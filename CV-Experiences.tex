%% Experiences section.
\section{\faBriefcase$\;$\textbf{Expériences professionnelles}}
\cventry{Mars 2015 $\rightharpoonup$\\ Maintenant}{\href{http://www.visian.fr}{Visian}}{Lead Developper \& Responsable Fablab}{Nanterre (92)}{}{Développeur Team Leader chez Visian, la BU Innovation autour des objets connectés de \href{http://www.neurones-it.com}{Neurones IT}. Responsable de l'équipe des développeurs. Réalisation et développement de PoC en méthodes agiles avec divers IoT (beacons, wearables, lunettes connectées, cartes programmables, tags NFC...). Remontée de données vers différents Clouds for IoT (Microsoft Azure, IBM Bluemix, AWS IoT).
  \ifthenelse{\equal{\displayVisianMissions}{yes}}{
    \newline{}
    \underline{Missions}:
    \begin{itemize}%
      \renewcommand{\labelitemi}{\textcolor{modernBlue}{\circ}}
    \item Réalisation de divers projets liés aux objets connectés.
    \item Encadrement et formation de stagiaires.
    \item Formation aux diverses technologies IoT (plateformes, hardware, protocoles…).
    \item Étude et comparatif des différents Clouds IoT existants.
    \end{itemize}
  }{}}

\cventry{Septembre 2014 $\rightharpoonup$\\ Mars 2015}{\href{http://www.groupe-sii.com/fr/}{Groupe SII}}{Assistant Responsable Innovation}{Paris (75)}{durée 6 mois, 3 jours/sem}{Étude et exploration des drones Open Source tels que le projet \href{http://www.bitcraze.se/}{Crazyflie}. Réalisation d’un prototype d'objet connecté à partir d’un \href{http://raspberrypicomputer.com/}{Raspberry Pi} et de divers capteurs (température, pression, GPS). Développement avec le SDK Java 8 Embedded ME for IoT, et remontée des données vers un serveur XMPP.
  \ifthenelse{\equal{\displaySIIMissions}{yes}}{
    \newline{}
    \underline{Missions}:
    \begin{itemize}%
      \renewcommand{\labelitemi}{\textcolor{modernBlue}{\circ}}
    \item Étude mécanique et électronique d'un drone.
    \item Sourcing de capteurs pour le Raspberry Pi.
    \item Réalisation d'un PoC de station météo mobile.
    \item Comparatif des protocoles IoT (XMPP, MQTT, AMQP).
    \end{itemize}
  }{}}

\cventry{Avril 2013 $\rightharpoonup$\\ Août 2013}{\href{http://www.thalesgroup.fr/}{Thales Services}}{Stagiaire Administrateur Système}{Ramonville (31)}{durée 5 mois}{Responsable de l'étude, de l'installation et du déploiement d'\href{http://www.openstack.org}{OpenStack} (cloud computing) dans un environnement de tests. Rédaction du dossier d’architecture matérielle, réalisation des scripts de déploiement automatique, tests et validations.
  \ifthenelse{\equal{\displayThalesMissions}{yes}}{
    \newline{}
    \underline{Missions}:
    \begin{itemize}%
      \renewcommand{\labelitemi}{\textcolor{modernBlue}{\circ}}
    \item Comparatif entre deux cloud IaaS: CloudStack et OpenStack.
    \item Déploiement d'OpenStack avec Mirantis Fuel.
    \item Rédaction de la documentation technique.
    \item Formation des utilisateurs.
    \end{itemize}
  }{}}
  
\cventry{Octobre 2012 $\rightharpoonup$\\ Avril 2013}{\href{http://www.novacom-services.com/}{Novacom Services}}{Stagiaire Développeur}{Ramonville (31)}{durée 7 mois, 2 jours/semaine}{Développement d'un jeu de données et d'un ensemble de tests unitaires et fonctionnels sur la couche service des applications, permettant l’automatisation des validations logicielles et ainsi garantir la non régression pendant les développements.
  \ifthenelse{\equal{\displayNovacomMissions}{yes}}{
    \newline{}
    \underline{Missions}:
    \begin{itemize}%
      \renewcommand{\labelitemi}{\textcolor{modernBlue}{\circ}}
    \item Formation au Java, Maven et JUnit.
    \item Correction et mises à jours des tests existants.
    \item Écriture de nouveaux tests unitaires.
    \item Documentation et Javadoc.
    \end{itemize}
  }{}}
