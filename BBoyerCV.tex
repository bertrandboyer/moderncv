% !TEX encoding = UTF-8 Unicode
% !TEX TS-program = LuaLaTeX

%% ModernCV.
\documentclass[10pt, a4paper]{moderncv}
\usepackage{luatextra}
\usepackage[french]{babel}
\usepackage[scale=0.85]{geometry}
\usepackage{fontawesome}
\usepackage{wrapfig}

%% ModernCV document style.
\moderncvstyle{classic}                         %% Styles: classic, casual, oldstyle, banking, fancy.
\moderncvcolor{blue}                            %% Colors: blue, orange, green, red, purple, grey, black.
\moderncvicons{awesome}                         %% Icons: marvosym, awesome, letters (banking only).
\renewcommand{\familydefault}{\sfdefault}
\nopagenumbers{}

%% Main colors of moderncv template.
\definecolor{modernBlue}{rgb}{0.22,0.45,0.70}   % Light blue.
\definecolor{modernGreen}{rgb}{0.35,0.70,0.30}  % Green.
\definecolor{modernGrey}{rgb}{0.55,0.55,0.55}   % Dark grey.
\definecolor{modernGrey2}{rgb}{0.45,0.45,0.45}  % Dark grey 2.
\definecolor{modernGrey3}{rgb}{0.35,0.35,0.35}  % Dark grey 3.
\definecolor{modernOrange}{rgb}{0.95,0.55,0.15} % Orange.
\definecolor{modernPurple}{rgb}{0.50,0.33,0.80} % Purple.
\definecolor{modernRed}{rgb}{0.95,0.20,0.20}    % Red.

%% Personnal informations.
\name{\textbf{Bertrand}}{\textbf{Boyer}}
\title{Ingénieur Innovation\newline{}Développeur C/C++/Java}
\address{}{}{}
\phone[mobile]{+33 (0) 6 10 04 27 16}
%% \phone[fixed]{}
%% \phone[fax]{}
\email{boyer.bertrand@gmail.com}
%% \homepage{}
\social[linkedin]{BertrandBoyer}
\social[github]{BertrandBoyer}
%% \social[twitter]{}
%% \photo[64pt][0.5pt]{}
\extrainfo{Diplômé Epitech Promotion 2015}
%% \quote{}

\begin{document}

%% Header.
\makecvtitle

%% Education section.
\section{\faGraduationCap$\;$\textbf{Parcours scolaire \& diplômes}}
\cventry{2014 $\rightharpoonup$ 2015}{\href{http://www.epitech.eu/paris/ecole-informatique-paris.aspx}{Epitech Paris}}{Cycle Master}{Paris (94)}{}{Seconde année de Cycle Master. Diplôme d'Expert en Technologies de l'information.}

\cventry{2013 $\rightharpoonup$ 2014}{\href{http://www.gcd.ie}{Griffith College Dublin}}{Cycle Master}{Dublin (Irlande)}{}{Première année de Master en Computing option Business et Management.}

\cventry{2011 $\rightharpoonup$ 2013}{\href{http://www.epitech.eu/toulouse/ecole-informatique-toulouse.aspx}{Epitech Toulouse}}{Cycle Bachelor}{Toulouse (31)}{}{Entrée directe en seconde année à Epitech (Tek2ED).}

\cventry{2008 $\rightharpoonup$ 2011}{\href{http://www.eisti.fr/}{EISTI}}{Classes préparatoires Maths Sup/Spé}{Pau (64)}{}{Obtention par équivalence d'une L1 Maths/Info.}

\cventry{2008}{\href{http://www.lyc-vauvenargues.ac-aix-marseille.fr/spip-lyc-lp/}{Lycée Vauvenargues}}{Baccalauréat Scientifique}{Aix-en-Provence (13)}{}{Obtention du Bac S options Mathématiques et Sciences de l'Ingénieur (SI).}

%% Experiences section.
\section{\faBriefcase$\;$\textbf{Expériences professionnelles}}
\cventry{Mars 2015 $\rightharpoonup$\\ Maintenant}{\href{http://www.visian.fr}{Visian}}{Lead Developper \& Responsable Fablab}{Nanterre (92)}{}{Développeur Team Leader chez Visian, la BU Innovation autour des objets connectés de \href{http://www.neurones-it.com}{Neurones IT}. Responsable de l'équipe des développeurs. Réalisation et développement de PoC en méthodes agiles avec divers IoT (beacons, wearables, lunettes connectées, cartes programmables, tags NFC...). Remontée de données vers différents Clouds for IoT (Microsoft Azure, IBM Bluemix, AWS IoT).\newline{}
  \underline{Missions}:
  \begin{itemize}%
    \renewcommand{\labelitemi}{\textcolor{modernBlue}{\circ}}
  \item Réalisation de divers projets liés aux objets connectés.
  \item Encadrement et formation de stagiaires.
  \item Formation aux diverses technologies IoT (plateformes, hardware, protocoles…).
  \item Étude et comparatif des différents Clouds IoT existants.
\end{itemize}}

\cventry{Septembre 2014 $\rightharpoonup$\\ Mars 2015}{\href{http://www.groupe-sii.com/fr/}{Groupe SII}}{Assistant Responsable Innovation}{Paris (75)}{durée 6 mois, 3 jours/sem}{Étude et exploration des drones Open Source tels que le projet \href{http://www.bitcraze.se/}{Crazyflie}. Réalisation d’un prototype d'objet connecté à partir d’un \href{http://raspberrypicomputer.com/}{Raspberry Pi} et de divers capteurs (température, pression, GPS). Développement avec le SDK Java 8 Embedded ME for IoT, et remontée des données vers un serveur XMPP.\newline{}
  \underline{Missions}:
  \begin{itemize}%
    \renewcommand{\labelitemi}{\textcolor{modernBlue}{\circ}}
  \item Étude mécanique et électronique d'un drone.
  \item Sourcing de capteurs pour le Raspberry Pi.
  \item Réalisation d'un PoC de station météo mobile.
  \item Comparatif des protocoles IoT (XMPP, MQTT, AMQP).
\end{itemize}}

\cventry{Avril 2013 $\rightharpoonup$\\ Août 2013}{\href{http://www.thalesgroup.fr/}{Thales Services}}{Stagiaire Administrateur Système}{Ramonville (31)}{durée 5 mois}{Responsable de l'étude, de l'installation et du déploiement d'\href{http://www.openstack.org}{OpenStack} (cloud computing) dans un environnement de tests. Rédaction du dossier d’architecture matérielle, réalisation des scripts de déploiement automatique, tests et validations.\newline{}
  \underline{Missions}:
  \begin{itemize}%
    \renewcommand{\labelitemi}{\textcolor{modernBlue}{\circ}}
  \item Comparatif entre deux cloud IaaS: CloudStack et OpenStack.
  \item Déploiement d'OpenStack avec Mirantis Fuel.
  \item Rédaction de la documentation technique.
  \item Formation des utilisateurs.
\end{itemize}}

\cventry{Octobre 2012 $\rightharpoonup$\\ Avril 2013}{\href{http://www.novacom-services.com/}{Novacom Services}}{Stagiaire Développeur}{Ramonville (31)}{durée 7 mois, 2 jours/semaine}{Développement d'un jeu de données et d'un ensemble de tests unitaires et fonctionnels sur la couche service des applications, permettant l’automatisation des validations logicielles et ainsi garantir la non régression pendant les développements.\newline{}
        \underline{Missions}:
  \begin{itemize}%
    \renewcommand{\labelitemi}{\textcolor{modernBlue}{\circ}}
  \item Formation au Java, Maven et JUnit.
  \item Correction et mises à jours des tests existants.
  \item Écriture de nouveaux tests unitaires.
  \item Documentation et Javadoc.
\end{itemize}}

%% Skills section.
\section{\faCode$\;$\textbf{Compétences techniques}}
\begin{cvcolumns}
  \cvcolumn{\faCogs$\;$Programmation}{Langage C/C++, Java SE, Java ME.\\Android, Android Wear et iOS (Swift).\\Bases de Lisp/Scheme, d'ASM Intel x86\_64.}
  \cvcolumn{\faTerminal$\;$Systèmes}{Administration Linux/Unix et shell scripts.\\Virtualisation: VirtualBox, VMWare, KVM.\\Cloud: Windows Azure, IBM Bluemix, AWS.}
\end{cvcolumns}

\begin{cvcolumns}
  \cvcolumn{\faConnectdevelop$\;$IoT}{Beacons (iBeacon \& Eddystone).\\Smartwatches sous Android Wear.\\Raspberry Pi et capteur des prototypage.}
  \cvcolumn{\faWrench$\;$Outils \& méthodologies}{Visual Studio, Eclipse, XCode, Emacs.\\Subversion, Git, Maven, Makefile.\\Asana, Slack, méthodes agiles, TDD.}
\end{cvcolumns}

\begin{cvcolumns}
  \cvcolumn{\faGlobe$\;$Web}{Bases de Java EE et Hibernate.\\Connaissances en HTML/CSS, PHP.\\Notions de Javascript et NodeJS.}
  \cvcolumn{\faDesktop$\;$Logiciels \& bureautique}{Microsoft Visio, Microsoft Project, Gimp.\\\LaTeX/Beamer, OpenOffice, Microsoft Office.\\Travail sous Windows, Linux et Mac OS.}
\end{cvcolumns}

\begin{cvcolumns}
  \cvcolumn{\faDatabase$\;$Base de données}{Connaissances en BDD SQL (MySQL).\\Notions en BDD NoSQL (MongoDB).}
\end{cvcolumns}

%% Languages section.
\section{\faLanguage$\;$\textbf{Langues}}
\begin{cvcolumns}
  \cvcolumn{\textcolor{black}{Français:}$\;$\textcolor{modernBlue}{\faCircle$\;$\faCircle$\;$\faCircle$\;$\faCircle$\;$\faCircle}}{\textcolor{modernGrey3}{(Langue maternelle)}}
  \cvcolumn{\textcolor{black}{Anglais:}$\;$\textcolor{modernBlue}{\faCircle$\;$\faCircle$\;$\faCircle$\;$\faCircle$\;$\faCircleO}}{\textcolor{modernGrey3}{(TOEFL, Décembre 2012)}} %% E-TOEIC Epitech 810 points.
  \cvcolumn{\textcolor{black}{Espagnol:}$\;$\textcolor{modernBlue}{\faCircle$\;$\faCircle$\;$\faCircleO$\;$\faCircleO$\;$\faCircleO}}{\textcolor{modernGrey3}{(Niveau scolaire)}}
\end{cvcolumns}

%% Hobbies section.
\section{\faUserPlus$\;$\textbf{Autres Activités}}
\begin{cvcolumns}
  \cvcolumn{\faGroup$\;$Associatif}{%
    \begin{itemize}%
      \renewcommand{\labelitemi}{\textcolor{modernBlue}{\circ}}
    \item Scoutisme (en tant que scout et chef).
    \item Membre du Lab Robotique à Epitech.
    \item Photographe du BDE de l'EISTI.
    \item Responsable \textit{Cinéma \& Animation} à \href{http://air-eisti.fr}{Air-EISTI}.
  \end{itemize}}
  \cvcolumn{\faFilm$\;$Loisirs \& centres d'intérêts}{%
    \begin{itemize}%
      \renewcommand{\labelitemi}{\textcolor{modernBlue}{\circ}}
    \item Badminton, ski.
    \item Photographie, musique, cinéma.
    \item Lego, construction, robotique.
    \item Sciences et nouvelles technologies.
  \end{itemize}}
\end{cvcolumns}

\end{document}
